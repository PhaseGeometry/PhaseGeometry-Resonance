\documentclass[11pt,a4paper]{article}

\usepackage[utf8]{inputenc}
\usepackage[T1]{fontenc}
\usepackage[english]{babel}

\usepackage{amsmath,amssymb}
\usepackage{geometry}
\usepackage{hyperref}
\usepackage{titling} % для сдвига заголовка
\geometry{margin=1in}

\numberwithin{equation}{section}

% поднять шапку на ~1 см вверх
\setlength{\droptitle}{-1cm}

\title{Phase Geometry Series --- Part III\\[4pt]
Phase-Field Newtonian Gravity and Phase Clocks}

\author{Aleksey Turchanov}

\date{November 2025\\[0.3em]
\scriptsize Licensed under CC BY 4.0. See the Zenodo record (DOI: 10.5281/zenodo.17691420)}

\begin{document}

\maketitle

\begin{abstract}
We propose a simple phase-field model that reproduces standard Newtonian
gravity while tying the gravitational potential to the gradient energy of
a static phase field. The construction is deliberately modest: it is
non-relativistic, static, and restricted to the weak-field regime, but it
is mathematically consistent and compatible with the usual Newtonian
picture. Any static phase configuration $\varphi(\mathbf x)$ generates an
effective mass density $\rho_{\mathrm{eff}}(\mathbf x)\propto
|\nabla\varphi(\mathbf x)|^2$, which acts as a source for the Newtonian
potential $\Phi(\mathbf x)$ through the Poisson equation. Localised phase
defects then behave as effective gravitating masses with a $1/r$
far-field potential.

We further connect this phase-field gravity to phase-based clocks
(Josephson clocks) by embedding the potential $\Phi(\mathbf x)$ into the
weak-field relation for the metric component $g_{00}$ and the corresponding
gravitational redshift of clock frequencies. In this way, the same phase
field that sources the gravitational potential also controls the local
ticking rate of phase clocks. The result is a simple and transparent
``Newtonian floor'' for phase geometry: standard Newtonian dynamics and
weak-field time dilation expressed entirely in terms of a static phase
texture.

The phase-field coupling constant $\kappa$ introduced here is a
phenomenological parameter that measures how strongly phase-gradient
energy sources an effective Newtonian potential. It is \emph{distinct
from} the dimensionless phase--flux coefficient $\alpha(d,\omega)$ used
in Parts~I and~I-B of the series for superconducting phase rotators.
\end{abstract}


\tableofcontents

\section{Introduction}

The idea that geometry and phase are connected appears in many parts of
physics: in superconductivity and superfluidity, in Berry phases, and in
analogue gravity. In this series of works on phase geometry we explore
this connection systematically. Part~I discussed resonant field symmetry
in superconductors, and Part~II introduced phase-based clocks (Josephson
clocks) in weak gravitational fields using a five-dimensional
phase-fibre geometry.

This work is Part~III of the Phase Geometry Series:
\emph{Superconductivity and Weak Gravity}, which develops a unified
phase-based view of resonant field symmetry, clocks and gravity. The
conceptual foundations are outlined in the Series Overview, while
Part~I discusses resonant field symmetry and Part~I-B develops a
calculational framework for phase rotators and the phase--flux
coefficient $\alpha(d,\omega)$ in thick SNS weak links. Part~II then
introduces phase clocks (Josephson clocks) in weak gravitational fields.

Here, in Part~III, we focus on a phenomenological phase-field model of
Newtonian gravity and its coupling to phase clocks. We deliberately stay
away from strong fields and full general relativity. Instead, we
construct a phase-field Newtonian gravity in which:
\begin{itemize}
  \item a static phase field $\varphi(\mathbf x)$ carries gradient energy;
  \item this energy density plays the role of an effective mass density
    $\rho_{\mathrm{eff}}(\mathbf x)$;
  \item the Newtonian potential $\Phi(\mathbf x)$ obeys the usual Poisson
    equation with $\rho_{\mathrm{eff}}$ as a source;
  \item test particles move in $\Phi(\mathbf x)$ according to standard
    Newtonian mechanics;
  \item phase clocks (such as Josephson clocks) experience gravitational
    redshift determined by $\Phi(\mathbf x)$ in the weak-field limit.
\end{itemize}

The outcome is a minimal, internally consistent model: it does not
compete with general relativity, but it provides a clear ``playground''
where both gravity and clock rates are determined by a single static
phase field.

\section{Phase field and effective mass density}
\label{sec:phase_mass}

We consider a real scalar phase field $\varphi(\mathbf x)$ defined on
three-dimensional space. In this work we restrict to static
configurations, so $\varphi$ depends only on position $\mathbf x$, not on
time.

The basic assumption is that the gradient energy of $\varphi$ acts as a
source of gravity. We introduce an effective mass density
\begin{equation}
  \rho_{\mathrm{eff}}(\mathbf x)
  = \frac{\kappa}{c^2}\,|\nabla\varphi(\mathbf x)|^2,
  \label{eq:rho_eff_def}
\end{equation}
where $c$ is the speed of light (used to convert energy density to mass
density) and $\kappa$ is a phenomenological coupling constant. The
quantity $|\nabla\varphi|^2 = \nabla\varphi\cdot\nabla\varphi$ is the
squared gradient of the phase.

We then postulate that the Newtonian potential $\Phi(\mathbf x)$
satisfies the standard Poisson equation
\begin{equation}
  \nabla^2\Phi(\mathbf x) = 4\pi G\,\rho_{\mathrm{eff}}(\mathbf x),
  \label{eq:poisson}
\end{equation}
with $G$ the Newtonian gravitational constant and $\nabla^2$ the
Laplacian in three-dimensional Euclidean space.

Equation~\eqref{eq:poisson} is structurally identical to the usual
Newtonian field equation for gravity, with $\rho_{\mathrm{eff}}$ playing
the role of the mass density. The novelty is only in the definition
\eqref{eq:rho_eff_def} of $\rho_{\mathrm{eff}}$ in terms of the
phase-field gradient. The parameter $\kappa$ controls how strongly a
given phase-texture sources the effective Newtonian potential.

\section{General solutions and superposition}

Given $\rho_{\mathrm{eff}}(\mathbf x)$, the solution to the Poisson
equation~\eqref{eq:poisson} with boundary condition $\Phi(\mathbf x)\to
0$ as $|\mathbf x|\to\infty$ is the standard Newtonian potential
\begin{equation}
  \Phi(\mathbf x) = -G \int
  \frac{\rho_{\mathrm{eff}}(\mathbf x')}{|\mathbf x - \mathbf x'|}
  \,d^3x'.
  \label{eq:Phi_rho}
\end{equation}
Using \eqref{eq:rho_eff_def}, we can write this directly as a functional
of the phase field:
\begin{equation}
  \Phi(\mathbf x) = -\,\frac{G\kappa}{c^2} \int
  \frac{|\nabla\varphi(\mathbf x')|^2}{|\mathbf x - \mathbf x'|}
  \,d^3x'.
  \label{eq:Phi_varphi}
\end{equation}
This expression states that the gravitational potential at point
$\mathbf x$ is determined by the full spatial distribution of the
phase-gradient energy.

Although $\rho_{\mathrm{eff}}$ is quadratic in $\nabla\varphi$ and
therefore nonlinear in $\varphi$, the Poisson equation is linear in
$\Phi$. As a result, superposition holds at the level of the potential.
If we decompose the phase field into a sum of contributions,
\begin{equation}
  \varphi(\mathbf x) \approx \sum_i \varphi_i(\mathbf x),
\end{equation}
then
\begin{equation}
  \rho_{\mathrm{eff}}(\mathbf x) =
  \frac{\kappa}{c^2}\,|\nabla\varphi(\mathbf x)|^2
  = \frac{\kappa}{c^2}\,
    \Bigl|\sum_i \nabla\varphi_i(\mathbf x)\Bigr|^2,
\end{equation}
and the individual contributions to $\Phi(\mathbf x)$ can still be
superposed via~\eqref{eq:Phi_rho}, even though $\rho_{\mathrm{eff}}$ is
not simply a sum of separate densities.

\section{Effective mass of a localized phase defect}
\label{sec:Meff}

We are particularly interested in localized phase structures, or
\emph{phase defects}, whose gradient energy is concentrated in a finite
region of space. For such a defect, the effective mass is defined as
\begin{equation}
  M_{\mathrm{eff}} = \int \rho_{\mathrm{eff}}(\mathbf x)\,d^3x
  = \frac{\kappa}{c^2}
    \int |\nabla\varphi(\mathbf x)|^2\,d^3x.
  \label{eq:Meff_def}
\end{equation}
If the defect is localized such that $\rho_{\mathrm{eff}}$ vanishes
sufficiently quickly at large $|\mathbf x|$, then for distances
$r = |\mathbf x|$ much larger than the size of the defect, the potential
takes the familiar form
\begin{equation}
  \Phi(r) \approx -\,\frac{G M_{\mathrm{eff}}}{r},
\end{equation}
up to corrections of order $R/r$, where $R$ is a characteristic size of
the defect. Thus, a localized phase defect behaves as an effective
gravitating mass, with $M_{\mathrm{eff}}$ determined by the integrated
phase-gradient energy.

In this sense, the phase-field model provides a concrete realisation of
the idea that phase textures can mimic point-like gravitating objects at
long distances, while retaining a smooth internal structure.

\section{Cylindrical phase defect: qualitative example}

To illustrate the construction, consider a simple cylindrical defect
centred on the $z$-axis. In cylindrical coordinates $(r,\theta,z)$ we
assume translational symmetry along $z$ and rotational symmetry in
$\theta$, so that $\varphi$ depends only on $r$:
\begin{equation}
  \varphi = \varphi(r).
\end{equation}
We further assume that $\varphi(r)$ interpolates between two constant
values over a radial scale $R$, with negligible gradients outside a
finite region.

The gradient of the phase is radial, and we have
\begin{equation}
  |\nabla\varphi|^2 = \left(\frac{d\varphi}{dr}\right)^2.
\end{equation}
The effective mass per unit length along the $z$-axis is then
\begin{equation}
  \frac{M_{\mathrm{eff}}}{L_z}
  = \int \rho_{\mathrm{eff}}(\mathbf x)\,d^2x_\perp
  = \frac{\kappa}{c^2} \int
    \left(\frac{d\varphi}{dr}\right)^2 2\pi r\,dr,
\end{equation}
where the integral is over the transverse plane. For a simple profile
with characteristic gradient
$|\mathrm d\varphi/\mathrm dr|\sim \Delta\varphi/R$ over a radial shell
of width $\sim R$, we obtain the order-of-magnitude estimate
\begin{equation}
  \frac{M_{\mathrm{eff}}}{L_z}
  \sim \frac{\kappa}{c^2}\,2\pi R
  \left(\frac{\Delta\varphi}{R}\right)^2 R
  \sim \frac{2\pi\kappa}{c^2}\,\frac{(\Delta\varphi)^2}{R}.
\end{equation}
Thus, for fixed phase contrast $\Delta\varphi$, narrower defects carry
more effective mass per unit length, reflecting the higher concentration
of gradient energy.

This cylindrical example is only qualitative, but it shows how specific
phase profiles map to effective mass distributions in the present
framework.

\section{Static Lagrangian formulation}

The phase-field model of Newtonian gravity can be encoded in a static
Lagrangian (or, more precisely, an energy functional). We consider a
functional of the fields $\varphi(\mathbf x)$ and $\Phi(\mathbf x)$:
\begin{equation}
  \mathcal{E}[\varphi,\Phi]
  = \int d^3x\left[
      \frac{\kappa}{2c^2}|\nabla\varphi|^2
      + \frac{1}{8\pi G}|\nabla\Phi|^2
    \right].
  \label{eq:energy_functional}
\end{equation}
The first term represents the gradient energy of the phase field, while
the second is the usual field-energy of the Newtonian potential.

To impose the Poisson relation between $\Phi$ and $\rho_{\mathrm{eff}}$,
one can either:
\begin{itemize}
  \item treat \eqref{eq:energy_functional} as a functional of $\Phi$ only
  and use the standard variational principle $\delta\mathcal{E}/\delta\Phi
  = 0$ to obtain $\nabla^2\Phi = 0$ with sources encoded in boundary
  conditions; or
  \item introduce a Lagrange multiplier field to enforce the constraint
  \eqref{eq:poisson} explicitly.
\end{itemize}
For the purposes of this phenomenological model, it is usually simpler
to regard \eqref{eq:poisson} as the defining equation for $\Phi$ in
terms of $\varphi$, with \eqref{eq:energy_functional} providing a useful
way to think about the total energy stored in both the phase field and
the gravitational potential.

\section{Test-particle dynamics}

Once the potential $\Phi(\mathbf x)$ is known, the motion of a classical
test particle of mass $m$ is governed by the usual Newtonian dynamics.
The Lagrangian for a test particle with position $\mathbf x(t)$ is
\begin{equation}
  L_{\mathrm{particle}} =
  \frac{1}{2} m \dot{\mathbf x}^2 - m\Phi(\mathbf x),
\end{equation}
and the Euler--Lagrange equations yield
\begin{equation}
  m\ddot{\mathbf x} = -m\nabla\Phi(\mathbf x),
\end{equation}
or simply
\begin{equation}
  \ddot{\mathbf x} = -\nabla\Phi(\mathbf x).
\end{equation}
Thus, test particles move as they would in classical Newtonian gravity,
with the only twist being that the potential $\Phi(\mathbf x)$ is itself
sourced by the phase-gradient energy according to \eqref{eq:rho_eff_def}
and \eqref{eq:poisson}.

\section{Phase clocks and Josephson-based superconducting oscillators}

In Part~II of the Phase Geometry Series, phase-based clocks were
described within a five-dimensional phase-fibre geometry. There the key
result was a phase-rate law of the form
\begin{equation}
  \frac{d\phi}{dt} = \omega_0\sqrt{g_{00}(\mathbf x)},
\end{equation}
where $g_{00}$ is the time-time component of the metric, $\omega_0$ is a
reference frequency, and $\phi$ is an internal phase variable. In the
weak-field limit, with
\begin{equation}
  g_{00}(\mathbf x) \simeq 1 + \frac{2\Phi(\mathbf x)}{c^2},
\end{equation}
this becomes
\begin{equation}
  \frac{d\phi}{dt} \simeq \omega_0
  \left(1 + \frac{\Phi(\mathbf x)}{c^2}\right),
\end{equation}
which reproduces the familiar gravitational redshift:
clocks deeper in the potential well tick more slowly.

In the present phase-field Newtonian model, we take $\Phi(\mathbf x)$ to
be the potential generated by the phase-gradient energy of
$\varphi(\mathbf x)$. Phase-based clocks (including Josephson clocks)
placed at different positions in this potential therefore experience
different ticking rates
\begin{equation}
  \omega(\mathbf x) \simeq \omega_0
  \left(1 + \frac{\Phi(\mathbf x)}{c^2}\right),
\end{equation}
with the fractional frequency shift
\begin{equation}
  \frac{\Delta\omega}{\omega_0}
  \simeq \frac{\Delta\Phi}{c^2},
\end{equation}
where $\Delta\Phi$ is the potential difference between two locations.

The superconducting implementations developed in Parts~I and~I-B
provide concrete examples of phase clocks: Josephson oscillators and
phase rotators whose frequencies and phases can, in principle, be
measured with high precision. In that context, the present Part~III can
be viewed as a phenomenological model for how engineered phase textures
could mimic weak gravitational wells and redshift these clock
frequencies.

\section{Limits and domain of validity}

The phase-field Newtonian model constructed here is deliberately modest
in scope. Its domain of validity is restricted by several assumptions:
\begin{itemize}
  \item \textbf{Static phase field.} The phase field $\varphi(\mathbf x)$
    is assumed to be time-independent. Allowing $\varphi(\mathbf x,t)$
    to vary in time would introduce additional terms in both the
    effective energy density and the metric, and is left for future
    work.
  \item \textbf{Weak-field regime.} The use of the Newtonian potential
    $\Phi(\mathbf x)$ and the weak-field approximation for $g_{00}$
    restrict the model to situations where $|\Phi|/c^2\ll 1$.
  \item \textbf{Non-relativistic test particles.} Test particles are
    treated within non-relativistic mechanics. Relativistic corrections
    can be incorporated, but are not considered here.
  \item \textbf{Phenomenological coupling.} The parameter $\kappa$ is
    phenomenological: it is not derived from an underlying microscopic
    theory, and may in principle depend on the physical system in which
    the phase field is realised. It is also independent of the
    superconducting phase--flux coefficient $\alpha(d,\omega)$ used in
    Parts~I and~I-B.
\end{itemize}
Within these limits, the Newtonian sector of the phase-field model is
well-defined and internally consistent: any static phase texture
$\varphi(\mathbf x)$ yields an effective mass density
$\rho_{\mathrm{eff}}(\mathbf x)$, a Newtonian potential $\Phi(\mathbf x)$
with the usual $1/r$ far-field behaviour for localised defects, and
standard test-particle dynamics and clock redshift in that potential.

\section{Discussion and outlook}

We have constructed a simple phase-field model of Newtonian gravity in
which:
\begin{itemize}
  \item the effective mass density is given by the gradient energy of a
    static phase field,
    $\rho_{\mathrm{eff}} = (\kappa/c^2)|\nabla\varphi|^2$;
  \item the Newtonian potential obeys the usual Poisson equation with
    this source;
  \item localised phase defects carry a finite effective mass
    $M_{\mathrm{eff}}$ and generate a $1/r$ far-field potential
    $\Phi(r) = -GM_{\mathrm{eff}}/r$;
  \item test particles move as they would in classical Newtonian
    gravity;
  \item phase-based clocks experience gravitational redshift determined
    by the same potential $\Phi(\mathbf x)$, which is itself a
    functional of the phase texture.
\end{itemize}

On this basis, one can say that the ``Newtonian floor'' of phase
geometry is complete: the standard physics of slow bodies and weak-field
time dilation has been re-expressed in terms of a single static phase
field. The coupling constant $\kappa$ encapsulates how strongly
phase-gradient energy is allowed to act as an effective source for
Newtonian curvature.

Although deliberately modest in scope, the present phase-field
construction has several useful consequences. First, it provides a
single phase-based description of effective mass, Newtonian potential,
and clock rates: the same static phase texture determines the effective
mass density, the gravitational potential $\Phi(\mathbf x)$, and the
local ticking of phase clocks via $g_{00}(\mathbf x)$. This offers a
simple and conceptually transparent way to view weak gravity and time
dilation through the lens of phase geometry.

Second, the model can serve as a theoretical basis for analogue gravity
systems in superconductors and other phase-ordered media. By
engineering phase textures (defects, arrays, patterned structures), one
can in principle design effective gravitational potentials and study the
motion of quasiparticles and clocks in those backgrounds, including
lensing-like effects and potential horizon analogues.

Third, the static Newtonian framework developed here provides a starting
point for dynamical generalisations. Allowing the phase field
$\varphi(\mathbf x,t)$ to become time dependent would induce
fluctuations of the effective potential and $g_{00}$, opening a route
towards phase-field analogues of gravitational waves and other
time-dependent gravitational phenomena in controllable laboratory
settings.

In this sense the phase-field construction is best viewed as a
convenient playground for toy and analogue models of gravity, rather
than as a competitor to general relativity. Its Newtonian, phase-based
formulation makes it particularly suitable for designing and analysing
controllable ``mini-gravity'' scenarios in superconductors and other
phase-ordered media where the underlying phase field is directly
accessible and, at least in principle, tunable.

\section*{Acknowledgements}

The author thanks colleagues and collaborators for discussions on phase
geometry, analogue gravity and phase-based clocks.

\begin{thebibliography}{9}

\bibitem{PartI}
A.~Turchanov,
\newblock \emph{Resonant Field Symmetry in Superconductors:
A Standing-Wave Picture of Meissner Screening and Josephson Barriers},
Phase Geometry Series --- Part I, preprint (2025).

\bibitem{PartIB}
A.~Turchanov,
\newblock \emph{Resonant Field Framework and Classical Basis for Magnetic
Phase Control of a Thick SNS Weak Link},
Phase Geometry Series --- Part I-B, preprint (2025).

\bibitem{PartII}
A.~Turchanov,
\newblock \emph{Phase-Coherent Josephson Devices as Clocks in Weak
Gravitational Fields},
Phase Geometry Series --- Part II, preprint (2025).

\bibitem{Overview}
A.~Turchanov,
\newblock \emph{Phase Geometry Series --- Overview: Coherence,
Electromagnetism and Weak Gravity in a Phase-Based Framework},
Zenodo (2025).

\end{thebibliography}

\end{document}
