\documentclass[11pt,a4paper]{article}
\pdfoutput=1

\usepackage[utf8]{inputenc}
\usepackage[T1]{fontenc}
\usepackage[english]{babel}

\usepackage{amsmath,amssymb}
\usepackage{bm}
\usepackage{geometry}
\usepackage{hyperref}
\usepackage{enumitem}
\geometry{margin=1in}

\title{PhaseGeometry-Resonance Phase-Device Zoo:\\[4pt]
Towards a Unified Phase Description of Quantum and Semiclassical Devices}

\author{Aleksey Turchanov}

\date{December 2025}

\begin{document}

\maketitle

\begin{center}
\small
Licensed under Creative Commons Attribution 4.0 International (CC BY 4.0).\\[2pt]
Zenodo DOI: 10.5281/zenodo.17807163
\end{center}

\vspace{1em}

\begin{abstract}
In standard device physics, tunnelling, interference, negative differential
resistance and charge quantisation are usually treated within a collection of
separate models: Esaki diodes, resonant-tunnelling diodes, Gunn and IMPATT
oscillators, Josephson junctions, SQUIDs, Aharonov--Bohm rings, quantum point
contacts, single-electron transistors, quantum cascade lasers, and many
variants. Within the PhaseGeometry programme it is natural to repackage these
elements as members of a single ``phase-device zoo'', each realising a
particular way in which a phase field controls transport. This short note
provides a compact taxonomy of such devices and sketches a unifying phase
description that can be linked both to the Z$_2$ phase medium and to the 5D
phase-fibre picture.
\end{abstract}

\section{Motivation}

The PhaseGeometry framework is built around the idea that many apparently
different physical systems can be understood as realisations of the same
underlying phase structures. In the Z$_2$ programme the emphasis is on a
binary phase medium and its rôle in cosmology and quantum foundations; in the
5D phase-fibre programme, phase is promoted to a genuine geometric coordinate
$\phi$ attached to each spacetime point. In both cases, phase is not a mere
auxiliary label but a dynamical object which can store, transport and process
information.

From this point of view, a wide variety of condensed-matter and mesoscopic
devices can be treated as \emph{phase devices}: physical systems whose
functionality is governed by the behaviour of some phase field (or an
effective phase variable) and its coupling to currents, charges, and external
fields. The purpose of this note is twofold:
\begin{itemize}
  \item to collect a minimal ``phase-device zoo''---a list of representative
  devices which already play, or can naturally play, a rôle in PhaseGeometry;
  \item to outline simple unifying principles which allow one to describe
  these systems within a single phase-based language, without replacing
  standard microscopic models.
\end{itemize}

\section{Phase-device zoo}
\label{sec:zoo}

Instead of a wide table, we list representative devices as a structured
taxonomy. For each entry we indicate (i) the physical regime, (ii) the
dominant mechanism, (iii) the natural phase object, (iv) the presence or
absence of negative differential resistance (NDR), and (v) the suggested rôle
within the PhaseGeometry programme.

\setlist[description]{leftmargin=2.5em,labelindent=0em,font=\normalfont\bfseries}

\begin{description}

  \item[Esaki (tunnel) diode.]
  \emph{Regime:} semiconductor, heavily doped $p$--$n$ junction with a very
  thin depletion region. \emph{Mechanism:} direct tunnelling through the
  depletion barrier; overlap of filled and empty states on the two sides.
  \emph{Phase object:} Bloch-wave phase across the junction; can be treated as
  an effective normal-state phase rotator with coupling
  $\alpha_N(d,V)$. \emph{NDR:} yes (pronounced peak and valley in I--V).
  \emph{PhaseGeometry rôle:} normal-state prototype of a phase rotator;
  simplest NDR device to rephrase in the PhaseGeometry language.

  \item[Resonant tunnelling diode (RTD).]
  \emph{Regime:} semiconductor heterostructure with a double barrier and a
  quantum well. \emph{Mechanism:} resonant tunnelling via discrete levels in
  the well; current peaks when incoming energy matches a bound level.
  \emph{Phase object:} phase of standing waves in the well; resonant phase
  condition across a multi-barrier stack. \emph{NDR:} yes.
  \emph{PhaseGeometry rôle:} clean example of a \emph{resonant} phase rotator;
  normal-state analogue of Andreev / Josephson bound levels.

  \item[Gunn diode.]
  \emph{Regime:} bulk or heterostructure semiconductor with multi-valley
  conduction band. \emph{Mechanism:} intervalley transfer and formation of
  travelling high-field domains in the bulk. \emph{Phase object:} collective
  phase of a propagating space--charge / field domain. \emph{NDR:} yes
  (effective NDR and microwave generation). \emph{PhaseGeometry rôle:} example
  of NDR without tunnelling; macroscopic phase of a travelling pattern in the
  drift medium.

  \item[IMPATT / TRAPATT / BARITT diodes.]
  \emph{Regime:} semiconductor junctions and heterostructures operated at high
  fields and microwave frequencies. \emph{Mechanism:} impact ionisation plus
  finite transit time; phase delay between RF field and current.
  \emph{Phase object:} phase of the AC current relative to the RF field; a
  transit-time phase shift. \emph{NDR:} yes at RF/microwave frequencies.
  \emph{PhaseGeometry rôle:} high-frequency phase devices where negative
  resistance arises from controlled phase lags.

  \item[Josephson junction (SIS / SNS / Dayem bridge).]
  \emph{Regime:} superconductor--insulator--superconductor or
  superconductor--normal--superconductor weak link. \emph{Mechanism:}
  Cooper-pair tunnelling and proximity effect in the weak region.
  \emph{Phase object:} order-parameter phase difference $\Delta\phi$ across
  the weak link. \emph{NDR:} no in DC I--V (but strongly nonlinear and
  hysteretic in some regimes). \emph{PhaseGeometry rôle:} canonical
  superconducting phase rotator; base model for PhaseGeometry SNS/SIS devices
  and resonant field cells.

  \item[dc / rf SQUID.]
  \emph{Regime:} superconducting loop with one or two Josephson junctions.
  \emph{Mechanism:} flux--phase conversion and interference of Josephson
  currents around the loop. \emph{Phase object:} loop phase / enclosed flux;
  $\phi$ controlled by magnetic flux $\Phi$ through the loop.
  \emph{NDR:} no (used primarily as a sensitive sensor). \emph{PhaseGeometry
  rôle:} interferometric phase meter for the superconducting order parameter;
  bridge between local and global phase control.

  \item[Aharonov--Bohm ring.]
  \emph{Regime:} mesoscopic normal metal or 2DEG ring with coherent transport.
  \emph{Mechanism:} quantum interference of two paths enclosing magnetic flux.
  \emph{Phase object:} relative phase between paths; Aharonov--Bohm phase
  proportional to enclosed flux $\Phi$. \emph{NDR:} no (oscillatory
  conductance as a function of flux). \emph{PhaseGeometry rôle:} minimal
  normal-state interferometer; pure example of phase-controlled conductance
  without superconductivity.

  \item[Quantum point contact (QPC).]
  \emph{Regime:} narrow constriction in a 2DEG or nanowire supporting a small
  number of 1D modes. \emph{Mechanism:} mode quantisation and coherent
  transmission in 1D channels; conductance quantisation in steps of
  $2e^2/h$. \emph{Phase object:} scattering phase of each transport mode; phase
  of transmission amplitudes. \emph{NDR:} no (step-like conductance).
  \emph{PhaseGeometry rôle:} prototype of phase-dependent, quantised
  conductance; natural entry point for PhaseGeometry in mesoscopic transport.

  \item[Single-electron transistor (SET).]
  \emph{Regime:} small normal or superconducting island connected by tunnel
  barriers. \emph{Mechanism:} Coulomb blockade and charge quantisation;
  sequential tunnelling events. \emph{Phase object:} phase of charge states on
  the island (dual to superconducting phase); an effective phase in the charge
  basis. \emph{NDR:} not in simple DC I--V (Coulomb staircase rather than
  N-shaped characteristics). \emph{PhaseGeometry rôle:} charge--phase dual to
  Josephson devices; connects PhaseGeometry to discrete charge transport and
  island devices.

  \item[Quantum cascade laser (QCL).]
  \emph{Regime:} semiconductor superlattice with multiple wells and barriers.
  \emph{Mechanism:} cascaded resonant tunnelling and optical transitions along
  a ladder of engineered subbands. \emph{Phase object:} phase of envelope
  states across the superlattice; phase matching between tunnelling and
  optical emission. \emph{NDR:} no (functions as a gain medium).
  \emph{PhaseGeometry rôle:} multistage phase-engineered tunnelling structure;
  example of extended phase engineering in the energy domain.

\end{description}

\section{Unifying principles for phase devices}
\label{sec:unification}

The list above deliberately mixes conventional semiconductor devices,
superconducting elements and mesoscopic interferometers. The unifying idea is
that each of them can be associated with an effective phase variable and a
set of coupling coefficients which quantify how this phase responds to
external drives and how it feeds back into observable currents and voltages.

\subsection{Effective phase fields and phase rotators}

The first step is to identify, for each microscopic setting, a natural phase
object:
\begin{itemize}
  \item in normal and semiconductor structures (Esaki, RTD, QPC, QCL), the
  phase of Bloch or envelope states provides a coarse-grained phase
  $\phi_\mathrm{B}(x)$ across the active region;
  \item in superconducting devices (Josephson junctions, SQUIDs), the
  order-parameter phase $\phi_\mathrm{SC}(x)$ is already explicit and directly
  controls supercurrents;
  \item in systems with travelling domains or delayed currents (Gunn, IMPATT),
  one can define a collective phase $\phi_\mathrm{dom}(x,t)$ of the propagating
  pattern;
  \item in charge-quantised systems (SET), a conjugate relation between
  discrete charge and a phase-like variable $\phi_Q$ naturally appears.
\end{itemize}

Once such a phase variable is identified, the device can be treated as a
\emph{phase rotator}: a local region in which the phase changes by some amount
$\Delta\phi$ in response to external stimuli (bias voltage, magnetic flux,
microwave drive, injected current). The microscopic details enter through
effective coupling coefficients, schematically
\begin{equation}
  \Delta\phi \;=\; \alpha(d,V,\omega,B,\ldots)\,\times\,\text{(drive)} \,,
\end{equation}
with $d$ the characteristic thickness or length of the active region and
$(V,\omega,B,\ldots)$ denoting the applied voltage, frequency, magnetic field
and other control parameters. In superconducting SNS/SIS links this is the
$\alpha(d,\omega)$ already used in PhaseGeometry; for Esaki and RTD devices
one can introduce a normal-state analogue $\alpha_N(d,V,\omega)$ calibrated
to standard tunnelling models.

\subsection{Negative differential resistance as phase restructuring}

In devices with NDR (Esaki, RTD, Gunn, IMPATT), the same formal phase
description can be used to reinterpret the origin of the non-monotonic
current--voltage characteristic:
\begin{itemize}
  \item in tunnelling devices (Esaki, RTD), the overlap of filled and empty
  states selects an effective set of phases which contribute to transport; as
  $V$ is increased, the accessible phase channels reorganise, producing a
  peak and subsequent drop in current;
  \item in Gunn and IMPATT structures, the phase of domain motion or the
  phase lag between field and current crosses special values, leading to an
  effective negative resistance at certain frequencies.
\end{itemize}
In all these cases, NDR can be seen as a \emph{restructuring of phase flow}
through the device: a change in the pattern by which phase modes carry
current, rather than a purely local change of resistance.

\subsection{Interferometric phase meters}

Aharonov--Bohm rings, SQUIDs and related interferometers play a special rôle
as \emph{phase meters}. They do not primarily act as phase rotators, but as
devices that convert accumulated phase differences into measurable changes in
conductance or critical current. In the PhaseGeometry picture, such elements
link local phase dynamics to global constraints (e.g.\ flux quantisation in a
loop) and provide natural interfaces between abstract phase fields and
laboratory observables.



\section{Directions for further development}

The phase-device zoo suggests several concrete directions for the
PhaseGeometry programme:
\begin{enumerate}
  \item \textbf{Normal-state phase rotators.} Develop an explicit mapping
  between standard tunnelling models (Tsu--Esaki, NEGF) and effective
  normal-state phase rotators with coefficients $\alpha_N(d,V,\omega)$ for
  Esaki and resonant-tunnelling diodes.
  \item \textbf{Unified SNS / Esaki / RTD language.} Formulate a common
  phase-based description in which superconducting SNS/SIS junctions and
  normal tunnelling devices appear as limiting cases of a single
  phase-rotator template, differing only by the type of phase field and
  microscopic calibration of $\alpha$.
  \item \textbf{Phase-engineered NDR.} Use the phase-flow viewpoint to design
  new NDR structures where the current peak and valley are controlled by
  engineered phase spectra (multi-barrier stacks, superlattices, hybrid
  SC/normal configurations).
  \item \textbf{Charge--phase dual networks.} Extend the PhaseGeometry
  treatment to networks combining Josephson junctions, SETs and related
  devices, exploiting the duality between phase and charge to construct
  programmable phase media at the circuit level.
  \item \textbf{Bridging to Z$_2$ and 5D frameworks.} Interpret the effective
  device phases as coarse-grained manifestations of either (i) binary phase
  domains in the Z$_2$ medium, or (ii) trajectories in the 5D phase fibre
  $(x^\mu,\phi)$. This provides a route to embed concrete mesoscopic devices
  into the broader cosmological and foundational context of PhaseGeometry.
\end{enumerate}

\section{Status and future work}

The present note is intentionally conceptual and taxonomic. It collects
standard devices into a common phase language but does not re-derive their
transport properties from scratch within the Resonance formalism. In
particular, normal-state tunnelling devices (Esaki, RTD) and mesoscopic
interferometers (Aharonov--Bohm rings, QPCs) are treated at the level of
their dominant mechanisms and natural phase variables, with microscopic
details delegated to the standard condensed-matter literature.

Future work will include semi-technical mappings for selected devices, such
as Esaki and resonant-tunnelling diodes and Aharonov--Bohm rings, where
effective phase variables and normal-state coupling coefficients
$\alpha_N(d,V,\omega)$ are derived directly from tunnelling and interference
models. These worked examples will strengthen the link between the
PhaseGeometry Resonance language and existing NEGF, Usadel and related
approaches, and may be collected in a follow-up note.

\section{Summary}

This note does not attempt to replace any microscopic device model; rather, it
collects a set of standard quantum and semiclassical devices and repackages
them into a unified phase-based taxonomy. The key claim is modest but
structural: once an effective phase field is identified, a wide variety of
devices can be treated as phase rotators, phase meters, or phase-engineered
media. This provides a common language linking superconducting SNS/SIS
junctions, normal tunnelling devices, interferometers and charge-quantised
elements, and prepares the ground for embedding concrete device physics into
the Z$_2$ and 5D branches of the PhaseGeometry programme.

\end{document}
