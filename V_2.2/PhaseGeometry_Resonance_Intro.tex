\documentclass[11pt,a4paper]{article}
\pdfoutput=1

\usepackage[utf8]{inputenc}
\usepackage[T1]{fontenc}
\usepackage[english]{babel}

\usepackage{amsmath,amssymb}
\usepackage{\bm}
\usepackage{geometry}
\usepackage{hyperref}
\usepackage{enumitem}
\geometry{margin=1in}

\title{PhaseGeometry Resonance Programme:\\[4pt]
Operational Phase Framework for Superconducting and Quantum Devices}

\author{Aleksey Turchanov}

\date{December 2025}

\begin{document}

\maketitle

\begin{center}
\small
Licensed under Creative Commons Attribution 4.0 International (CC BY 4.0).\\[2pt]
Zenodo DOI: 10.5281/zenodo.17807163
\end{center}

\vspace{1em}


\begin{abstract}
The PhaseGeometry Resonance programme provides an operational phase-based
language for superconducting and quantum devices. Instead of treating each
device (Josephson junctions, SNS weak links, SQUIDs, tunnelling diodes,
mesoscopic interferometers) with a separate collection of models, the
Resonance framework describes them as phase rotators, phase meters and
resonant field cells characterised by a small set of effective parameters.
This note summarises the rôle of the Resonance layer within the broader
PhaseGeometry architecture and explains five concrete advantages of this
language: (i) unification and compression of device descriptions, (ii)
a clear figure of merit based on the phase--flux coupling coefficient
$\alpha(d,\omega)$, (iii) a clean interface to the Z$_2$ and Phase-Fibre
(5D) branches, (iv) direct experimental testability via thick SNS
phase-rotator devices, and (v) a conceptual shift from ``SQUID-only''
phase control to strongly localised phase rotators.
\end{abstract}

\section{Resonance within the PhaseGeometry architecture}

The PhaseGeometry programme consists of three main branches:
\begin{itemize}
  \item the \emph{Resonance} series, which introduces an operational
  phase-based description of superconducting and quantum devices in
  standard $3+1$-dimensional physics (Maxwell, London, Josephson,
  proximity effect);
  \item the \emph{Phase-Fibre} (5D) series, in which phase is promoted to
  a genuine geometric coordinate $\phi$ on a compact fibre, leading to
  a five-dimensional description of gravity, electromagnetism and phase
  clocks;
  \item the \emph{Z$_2$} series, where a binary phase medium provides a
  unified dark sector and quantum-foundational substrate.
\end{itemize}

In this architecture the Resonance layer plays a specific rôle:
\begin{itemize}
  \item it is formulated entirely in standard condensed-matter and
  electromagnetism language (no explicit Z$_2$ or 5D assumptions);
  \item it provides a compact operational dictionary for devices:
  superconductors are phase media, weak links are phase rotators,
  local standing-wave configurations of fields and currents are resonant
  field cells, and the phase--flux coupling is encoded in a single
  coefficient $\alpha(d,\omega)$;
  \item it is explicitly designed to be embeddable into both the Z$_2$
  and Phase-Fibre (5D) branches: once devices are written in this
  common phase language, they can be reinterpreted as effective
  manifestations of either a binary phase medium or a 5D phase fibre.
\end{itemize}

The aim of this note is to clarify \emph{why} such an intermediate
Resonance layer is useful and what concrete advantages it offers over
a collection of device-specific models.

\section{Advantage I: Unification and compression of device descriptions}
\label{sec:advantage_unification}

In conventional device physics each superconducting or mesoscopic element
typically comes with its own model:
\begin{itemize}
  \item Ginzburg--Landau and London theory for bulk superconductors;
  \item Usadel or Eilenberger equations for diffusive weak links;
  \item RCSJ-type models for Josephson junction dynamics;
  \item device-specific treatments for SQUIDs, resonators, hybrid
  structures, and so on.
\end{itemize}
While each model is well established, the conceptual landscape quickly
becomes fragmented. It is not obvious, for example, how to place a
thick SNS weak link, a Dayem bridge, an Esaki diode and an Aharonov--Bohm
ring into a single schematic framework without diving into their
microscopic details.

The Resonance framework instead operates with a deliberately small
set of phase-based objects:
\begin{itemize}
  \item a \emph{phase field} $\phi(\mathbf{r},t)$ representing the
  superconducting order parameter phase or an effective phase of a
  transport mode;
  \item a \emph{phase rotator}: a local region where the phase changes
  by some amount $\Delta\phi$ in response to an applied drive
  (voltage, current, magnetic flux, RF field);
  \item a \emph{resonant field cell}: a local standing-wave configuration
  of currents and fields associated with the phase rotator;
  \item a \emph{phase--flux coupling coefficient} $\alpha(d,\omega)$,
  which quantifies the efficiency with which external fields or flux
  modulate the phase across the rotator.
\end{itemize}

In this language:
\begin{itemize}
  \item a standard Josephson junction is a simple phase rotator with
  moderately weak coupling $\alpha$;
  \item a thick SNS weak link with an on-chip microcoil is a potentially
  \emph{strong} phase rotator with $|\alpha| \sim 0.3$--$0.6$;
  \item a SQUID loop with two junctions is a phase interferometer
  (phase meter) built from two rotators in a global flux quantisation
  condition;
  \item more exotic devices (Esaki diodes, resonant-tunnelling diodes,
  Aharonov--Bohm rings, quantum point contacts) can be added to the
  ``phase-device zoo'' as normal-state phase rotators or phase meters.
\end{itemize}

The key point is that many devices can be described \emph{schematically}
with the same few building blocks, and the microscopic diversity is
compressed into the calibration of $\alpha(d,\omega)$ and related
effective parameters.

\section{Advantage II: A clear figure of merit via $\alpha(d,\omega)$}
\label{sec:advantage_alpha}

In standard superconducting device discussions one often hears that a
given weak link is ``small'', ``moderately sensitive'' or ``strongly
coupled'' without a universal quantitative comparator. The Resonance
framework introduces a clean figure of merit:
\begin{equation}
  \alpha(d,\omega) \;=\;
  \frac{\Phi_0}{2\pi}\,
  \frac{\partial \phi_J}{\partial \Phi_{\text{ext}}}
  \bigg|_{d,\omega} \,,
\end{equation}
where $\phi_J$ is the phase difference across the weak link,
$\Phi_{\text{ext}}$ is an external magnetic flux (or more generally a
local field integrated over a suitable area), $\Phi_0$ is the flux
quantum, $d$ is a characteristic thickness or length of the active
region, and $\omega$ labels the drive frequency.

This dimensionless coefficient encapsulates how efficiently an external
field or flux drives the phase across the device:
\begin{itemize}
  \item $|\alpha| \ll 10^{-2}$ corresponds to a \emph{weak} phase
  rotator: the phase is only slightly modulated, and the device acts
  more like a passive element;
  \item $|\alpha| \sim 10^{-1}$--$1$ corresponds to a \emph{strong}
  phase rotator: local fields can produce large phase excursions and
  strong modulation of the supercurrent.
\end{itemize}

In the thick SNS with microcoil proposal developed in the Resonance
series, one finds that realistic geometries can achieve
$|\alpha| \sim 0.3$--$0.6$, comparable in phase efficiency to a
well-designed SQUID loop but in a compact, local geometry. The same
$\alpha(d,\omega)$ becomes a unifying figure of merit across different
devices, allowing direct comparison of how ``phase-active'' they are,
independent of microscopic details.

\section{Advantage III: Interface to Z$_2$ and Phase-Fibre (5D) branches}
\label{sec:advantage_interface}

The Z$_2$ and Phase-Fibre (5D) branches of PhaseGeometry are
interpretational and structural extensions:
\begin{itemize}
  \item the Z$_2$ programme treats the world as a binary phase medium
  with cosmological and quantum-foundational consequences;
  \item the Phase-Fibre programme promotes phase to a fifth coordinate
  $\phi$ on a compact fibre, giving a Kaluza--Klein-like unified
  picture of gravity, electromagnetism and phase clocks.
\end{itemize}

Both branches need a way to connect to \emph{real laboratory devices}.
The Resonance framework provides this bridge:
\begin{itemize}
  \item it packages devices into phase rotators, phase meters and
  resonant cells with well-defined phase fields and couplings
  $\alpha(d,\omega)$;
  \item once this is done, the same devices can be reinterpreted as
  probes of a binary Z$_2$ phase medium or as trajectories in a
  5D phase fibre;
  \item derivations of Josephson-like relations, gravitational
  redshift of phase clocks, and Sagnac-type phase shifts become
  statements about how the Z$_2$ medium or 5D geometry affects
  the phase rate and phase coupling in the operational layer.
\end{itemize}

In this sense the Resonance series is the \emph{shared device layer}
for both deeper branches: it keeps the device physics grounded in
standard SC/EM theory while making it immediately compatible with
the higher-level PhaseGeometry structures.

\section{Advantage IV: Direct experimental testability}
\label{sec:advantage_experimental}

A crucial strength of the Resonance framework is that it is not only
a conceptual language but also a basis for \emph{concrete experimental
proposals}. In particular, the thick SNS phase-rotator scheme with
a local microcoil provides:
\begin{itemize}
  \item a fully standard setup in terms of materials and fabrication:
  a superconducting strip, a $\sim 500$\,nm long diffusive SNS weak
  link, and a micron-scale on-chip spiral coil;
  \item a clear experimental protocol: drive the coil at microwave
  frequencies and compare the resulting Shapiro-step pattern (phase
  modulation induced by the coil) to that produced by a conventional
  RF drive applied to the junction;
  \item a direct extraction of $\alpha(d,\omega)$ from the relative
  height of magnetically driven and RF-driven Shapiro steps.
\end{itemize}

If the measured $\alpha(d,\omega)$ reaches the predicted
$|\alpha| \sim 0.3$--$0.6$ regime, this would confirm that strongly
localised phase rotators are experimentally feasible. If not, the
same framework provides a clear target for refining designs or revising
assumptions.

This experimental anchoring makes the Resonance layer \emph{falsifiable}
and practically relevant: it is not merely an abstract rephrasing, but
a way to design and test new classes of phase-sensitive devices.

\section{Advantage V: Conceptual shift beyond SQUID-only phase control}
\label{sec:advantage_conceptual}

In the traditional picture, strong magnetic control of superconducting
phase is typically associated with macroscopic loops:
\begin{itemize}
  \item SQUIDs, where the phase difference is tied to the total flux
  through the loop;
  \item extended superconducting rings in Sagnac-type configurations.
\end{itemize}

The Resonance framework reframes this intuition. By focusing on local
phase rotators and resonant field cells, it highlights that:
\begin{itemize}
  \item strong phase control does not require a large loop; it can be
  achieved in a compact, locally driven region (e.g.\ a thick SNS with
  microcoil), provided the geometry and frequency place the system in
  a high-$Q$, large-$\alpha$ regime;
  \item loops and rings (SQUIDs, superconducting interferometers) then
  appear as \emph{global constraints} on networks of local rotators,
  rather than the only way to access phase;
  \item the line between ``interferometric'' devices (SQUIDs, AB rings)
  and ``local'' devices (weak links, rotators) becomes more fluid:
  both are built from the same phase elements, arranged differently.
\end{itemize}

This conceptual shift is modest but important. It opens the door to:
\begin{itemize}
  \item designing new families of compact, strongly phase-sensitive
  devices that do not rely on large loops;
  \item thinking of circuits and lattices of phase rotators as
  programmable phase media, which is particularly relevant for both
  the Z$_2$ and Phase-Fibre branches;
  \item using the same language to compare and combine devices from
  superconducting, normal-state and mesoscopic domains within the
  phase-device zoo.
\end{itemize}

\section{Outlook}

The current Resonance framework already:
\begin{itemize}
  \item unifies superconducting weak links, SQUIDs and related devices
  as phase rotators and phase meters;
  \item provides a clear figure of merit $\alpha(d,\omega)$ for phase
  coupling strength;
  \item supports concrete experimental proposals, particularly thick
  SNS phase-rotator devices;
  \item interfaces cleanly with both the Z$_2$ and Phase-Fibre (5D)
  branches of PhaseGeometry.
\end{itemize}

Future work can extend this operational layer in several directions:
\begin{itemize}
  \item incorporating normal-state devices (Esaki, RTD, Gunn, IMPATT)
  and mesoscopic interferometers (Aharonov--Bohm rings, QPCs, SETs)
  into a fully developed phase-device zoo, calibrated via
  normal-state analogues of $\alpha(d,\omega)$;
  \item constructing networks of phase rotators and meters as
  programmable phase media, providing a circuit-level laboratory
  analogue of Z$_2$ and 5D phase structures;
  \item integrating more detailed microscopic calculations (Usadel,
  NEGF, microscopic Josephson theory) into the calibration of
  $\alpha(d,\omega)$ and related effective parameters, reinforcing the
  link between the Resonance layer and standard condensed-matter
  theory.
\end{itemize}

In summary, the Resonance programme is not a replacement for existing
microscopic models, but a compact operational layer that makes the
phase nature of devices explicit, quantifiable and transferable across
domains. It is this layer that allows PhaseGeometry to speak
simultaneously to experimental device physics and to its own deeper
Z$_2$ and Phase-Fibre structures.

\end{document}
