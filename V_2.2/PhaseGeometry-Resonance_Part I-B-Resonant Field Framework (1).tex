\documentclass[11pt,a4paper]{article}

\usepackage[utf8]{inputenc}
\usepackage[T1]{fontenc}
\usepackage[english]{babel}

\usepackage{amsmath,amssymb}
\usepackage{graphicx}
\usepackage{geometry}
\usepackage{booktabs}
\usepackage{array}
\usepackage{hyperref}

\geometry{margin=1in}

\numberwithin{equation}{section}

\title{Phase Geometry Series --- Part I-B\\[4pt]
Resonant Field Framework and Classical Basis\\
for Magnetic Phase Control of a Thick SNS Weak Link}

\author{Aleksey Turchanov}
\date{November 2025\\[0.3em]
\scriptsize Licensed under CC BY 4.0. Zenodo DOI: 10.5281/zenodo.17807163}


\begin{document}

\maketitle

\begin{abstract}
We develop a two-layer description of magnetic phase control in superconducting
weak links. On the effective level, we formulate a Resonant Field (RF) framework
in which a weak link is treated as a \emph{phase rotator}, characterized by a
dimensionless phase--flux coefficient $\alpha$ and surrounded by a localized
\emph{Resonant Field Cell}. On the microscopic level, we show how the same
objects and parameters arise from the standard theory of superconducting weak
links: Ginzburg--Landau / Usadel equations for the order parameter,
London--Maxwell electrodynamics for currents and fields, and the RCSJ model for
Josephson phase dynamics. We identify the normalized phase response
\begin{equation}
  \alpha = \frac{\Phi_0}{2\pi}\,
  \frac{\partial \varphi_J}{\partial \Phi_{\mathrm{ext}}}\,,
\end{equation}
and outline how thick SNS weak links with a local microcoil could reach the
``strong rotator'' regime $|\alpha|\sim 0.3$--$0.6$ under resonant conditions.
% <<< добавка про playground >>>
Although we focus on thick SNS weak links with a local microcoil as a concrete
experimental platform, the RF framework itself is not restricted to this
particular geometry and can, in principle, be applied to any weak-link structure
capable of supporting localized phase-field modes.

Within the broader Phase Geometry Series on superconductivity and weak gravity,
this work plays the role of \mbox{Part I-B}: it supplies the calculational backbone
that connects the conceptual Manifest I (Resonant Field Symmetry in
Superconductors) to concrete device proposals on the superconducting side and to
the phase-clock and gravity constructions developed in Parts II and III.
\end{abstract}

\pagebreak
\tableofcontents
\pagebreak
\section{Introduction}

Magnetic control of the Josephson phase is usually associated with macroscopic
loops such as dc SQUIDs, where the total phase bias is proportional to the
enclosed flux, $\varphi_J=(2\pi/\Phi_0)\Phi$. In contrast, a single weak link
addressed by a local microcoil is typically assumed to be almost insensitive to
magnetic flux: the phase only weakly responds to local fields because screening
currents in the leads suppress the effective phase--flux coupling.

In Manifest I of the Phase Geometry Series, we introduced the idea that
superconducting systems can host localized standing-field structures,
\emph{Resonant Field Cells}, which naturally act as \emph{phase rotators}. A
weak link that supports such a cell may then exhibit a much stronger local
phase response than na\"ive estimates suggest.

The purpose of the present Part I-B is three-fold:
\begin{itemize}
  \item to formulate an effective Resonant Field framework for magnetic phase
  control in weak links, introducing the notions of phase rotator, Resonant
  Field Cell, Phase--Control Equation and the key parameter $\alpha$;
  \item to show how the same objects and parameters arise from the classical
  microscopic description (Ginzburg--Landau / Usadel + London--Maxwell + RCSJ);
  \item to provide an explicit dictionary between the RF language and the
  classical formalism.
\end{itemize}

We focus on thick SNS weak links driven by an on-chip microcoil and ask whether
such structures can realistically reach a strong-rotator regime with
$|\alpha|\sim 0.3$--$0.6$.
\\
\\
\\
\section{Part I: Resonant Field Framework}

\subsection{System geometry and experimental goal}

We consider a superconducting strip interrupted by a central weak link of
thickness $d$ and lateral length $L_W$. A DC bias current $I_{DC}$ flows along
the strip, while an on-chip microcoil above the weak link carries an AC current
$I_{AC}(t)$ and generates a local magnetic field $B_{AC}(t)$ through a thin
insulating layer; see Fig.~\ref{fig:geometry}.

\begin{figure}[ht]
  \centering
  \includegraphics[width=0.85\textwidth]{1b-1.jpg}
  \caption{Geometry of a superconducting strip with a central thick weak link
  and an on-chip microcoil. A DC bias current $I_{DC}$ flows along the strip,
  while an AC current $I_{AC}$ in the microcoil generates a local magnetic
  field $B_{AC}(t)$ in the weak-link region through a thin insulating layer.}
  \label{fig:geometry}
\end{figure}

The experimental goal is to quantify how efficiently the local flux from the
microcoil can rotate the Josephson phase across the weak link, and how this
efficiency depends on the thickness $d$ and on dynamical effects such as
resonances.

\subsection{Phase field and phase flow}

In the RF language, the superconducting state is described by a phase field
$\phi(\mathbf r)$ defined over the strip, with supercurrent lines following
the gradients of this field. When a weak link is present, the phase drop
$\varphi_J$ across it becomes a key degree of freedom.

The RF framework focuses on how phase flow is redistributed in response to
local fields and how part of the energy may be stored in localized magnetic
structures rather than purely in longitudinal phase gradients.

\subsection{Phase rotator}

A \emph{phase rotator} is a localized element in which a relatively small
change in external magnetic flux produces a significant change in the phase
drop $\varphi_J$. Quantitatively, we introduce the phase--flux coefficient
$\alpha$ via
\begin{equation}
  \delta \varphi_J = \frac{2\pi}{\Phi_0}\,\alpha\,\delta \Phi_{\mathrm{ext}},
\end{equation}
so that $\alpha=1$ corresponds to the ideal SQUID limit.

Thin tunnel junctions typically realize $|\alpha|\ll 10^{-2}$ and are thus
\emph{weak rotators}. Thick SNS bridges, under appropriate conditions, may
become \emph{strong rotators} with $|\alpha|\sim 0.3$--$0.6$.

\subsection{Resonant Field Cell}

We define a \emph{Resonant Field Cell} as a localized standing-field structure
around the weak link in which phase, supercurrent and magnetic field form a
self-consistent resonant pattern. Conceptually, the cell plays the role of a
miniature cavity coupled to the weak link.

In thin junctions the RF Cell is absent or extremely rigid; in thick SNS
bridges it can be well-localized around the weak link and store a non-trivial
amount of magnetic tension. Figure~\ref{fig:rfcell} shows a qualitative sketch.

\begin{figure}[ht]
  \centering
  \includegraphics[width=0.8\textwidth]{1b-2.jpg}
  \caption{Qualitative picture of a Resonant Field Cell around a thick weak
  link. The shaded region marks the localized cell, and the opposite vertical
  arrows $B$ schematically indicate a standing magnetic-field pattern inside
  the cell.}
  \label{fig:rfcell}
\end{figure}

\subsection{Phase--flux coefficient $\alpha$}

Within the RF framework we treat the total phase drop across the junction as
\begin{equation}
  \varphi_J = \varphi_{\mathrm{bias}} + 
  \frac{2\pi}{\Phi_0}\,\alpha\,\Phi_{\mathrm{ext}},
\end{equation}
where $\varphi_{\mathrm{bias}}$ is the phase set by the DC current bias, and
$\Phi_{\mathrm{ext}}$ is the flux created by the microcoil.

The dimensionless coefficient $\alpha$ encodes how strongly the RF Cell around
the weak link couples the external flux to the Josephson phase. In the tunnel
limit $|\alpha|\ll 10^{-2}$; in the strong-rotator regime we aim at
$|\alpha|\sim 0.3$--$0.6$.

\subsection{Phase--Control Equation}

In the small-signal limit we can write the \emph{Phase--Control Equation} as
\begin{equation}
  \varphi_J(t) = \varphi_{\mathrm{bias}}(t) +
  \frac{2\pi}{\Phi_0}\,\alpha\,\Phi_{AC}(t),
\end{equation}
where $\Phi_{AC}(t)$ is the time-dependent coil flux. This effective equation
captures the net result of microscopic screening currents and localized field
patterns, and will later be derived from the RCSJ model in the classical
formalism.

\subsection{Static regimes: thin vs thick weak links}

\subsubsection{Thin barrier: negligible rotator}

For an SIS tunnel junction with $d\ll\xi$, the phase jump is concentrated in a
very narrow region, the local magnetic field is small, and the effective
phase--flux coupling is strongly screened. In RF terms:
\begin{itemize}
  \item the RF Cell is either absent or extremely rigid;
  \item $|\alpha(d)|\ll 10^{-2}$ for typical geometries;
  \item the weak link acts as a negligible rotator.
\end{itemize}

\subsubsection{Thick SNS bridge: emergence of a strong rotator}

When the thickness $d$ of an SNS weak link is comparable to the coherence
length $\xi$, the phase drop $\varphi_J$ is distributed over a wider region,
and a localized standing structure of currents and fields can appear around the
bridge. In RF language:
\begin{itemize}
  \item a well-defined RF Cell forms around the weak link;
  \item part of the energy is stored in local magnetic tension rather than purely
  longitudinal phase flow;
  \item the effective coupling $\alpha(d)$ can become significantly larger than
  in the tunnel limit.
\end{itemize}

Figure~\ref{fig:thin-thick} illustrates the contrast between thin and thick
regimes in terms of phase profiles and field patterns.

\begin{figure}[ht]
  \centering
  \includegraphics[width=0.85\textwidth]{1b-3.jpg}
  \caption{Qualitative comparison of phase profiles and field patterns for a
  thin tunnel junction (left) and a thick SNS bridge (right). In the tunnel
  limit the phase jump $\varphi(x)$ is sharply localized at the barrier and the
  local field structure is negligible; in the thick SNS regime the phase drop
  is distributed across the normal region and a localized RF Cell can form
  around the bridge.}
  \label{fig:thin-thick}
\end{figure}

\subsection{Dynamical regime and Shapiro steps}

In the presence of a DC bias $I_{DC}$ and an AC phase modulation, the Josephson
relation
\begin{equation}
  I(t) \approx I_c \sin\varphi_J(t)
\end{equation}
leads to Shapiro steps on the $I$--$V$ characteristic when the phase is locked
to an external frequency $\omega$.

Using the Phase--Control Equation with
\begin{equation}
  \Phi_{AC}(t) = \Phi_{0}^{(\mathrm{coil})}\sin\omega t,
\end{equation}
we obtain a phase modulation amplitude
\begin{equation}
  \Delta\varphi_{\mathrm{coil}} =
  \alpha\,\frac{2\pi}{\Phi_0}\,\Phi_{0}^{(\mathrm{coil})}.
\end{equation}
The heights of Shapiro steps depend on $\Delta\varphi_{\mathrm{coil}}$, so by
comparing coil-driven Shapiro steps with those produced by a calibrated RF
voltage drive, one can experimentally infer $\alpha$.

In the RF picture, a weak rotator ($|\alpha|\ll 10^{-2}$) will produce
negligible Shapiro steps under coil-only drive, while a strong rotator
($|\alpha|\sim 0.3$--$0.6$) will show clear steps even without direct RF
excitation of the junction.

\subsection{Qualitative prediction for $\alpha(d)$ and target regime}

The RF framework predicts a non-monotonic dependence of $|\alpha(d)|$ on the
thickness $d$ of the weak link:
\begin{itemize}
  \item for $d\ll\xi$: tunnel regime, $|\alpha(d)|\ll 10^{-2}$;
  \item for $d\sim\xi$: thick SNS regime, formation of a strong RF Cell, and a
  qualitative transition from negligible magnetic response to a strong rotator;
  \item for $d\gg\xi$: degradation of superconducting coupling, smearing of the
  RF Cell, and a reduction of $|\alpha(d)|$.
\end{itemize}

We introduce a target range for the strong-rotator regime:
\begin{equation}
  |\alpha_{\mathrm{target}}|\sim 0.3\text{--}0.6,\qquad d\sim\xi,
\end{equation}
corresponding to a $2$--$3$ orders-of-magnitude enhancement of the local phase
sensitivity compared to the tunnel limit. Figure~\ref{fig:alpha-d} shows a
qualitative sketch.

\begin{figure}[ht]
  \centering
  \includegraphics[width=0.75\textwidth]{1b-4.jpg}
  \caption{Qualitative sketch of $|\alpha(d)|$ versus normalized thickness
  $d/\xi$: small in the tunnel limit, peaking in the thick SNS regime, and
  decreasing for very large $d$. The shaded region indicates the target
  strong-rotator range $|\alpha|\sim 0.3$--$0.6$.}
  \label{fig:alpha-d}
\end{figure}

The detailed shape of $\alpha(d)$ must ultimately be obtained from microscopic
calculations (Part~II) or from experiments (e.g.\ thick SNS bridges with
microcoils). A simple order-of-magnitude estimate for realistic SNS bridges
and microcoils is given in Appendix~B.

\section{Part II: Classical Description of the Same System}

\subsection{Geometry and parameters}

We now turn to the standard classical description of the same system: a
superconducting strip with a central weak link and a local microcoil, as in
Fig.~\ref{fig:geometry}. The weak link may be:
\begin{itemize}
  \item an SIS tunnel junction with barrier thickness $d$;
  \item an SNS bridge where the normal layer has thickness $d$ and length
  $L_W$;
  \item a nanobridge with an effective normal region.
\end{itemize}

Material parameters include the coherence length $\xi$, London penetration
depth $\lambda_L$, normal resistance $R_N$ of the weak link, and capacitance
$C$.

\subsection{Order parameter, phase, and supercurrent}

The superconducting state is described by a complex order parameter
\begin{equation}
  \Psi(\mathbf r) = |\Psi(\mathbf r)|e^{i\theta(\mathbf r)},
\end{equation}
with phase $\theta(\mathbf r)$. In the GL/London approximation the supercurrent
density is
\begin{equation}
  \mathbf j_s(\mathbf r) =
  \frac{2e\hbar}{m^\ast}\,|\Psi(\mathbf r)|^2
  \left(
    \nabla\theta(\mathbf r)
    - \frac{2e}{\hbar}\mathbf A(\mathbf r)
  \right),
\end{equation}
where $\mathbf A$ is the vector potential and $\mathbf B=\nabla\times\mathbf A$.

\subsection{Weak link and Josephson phase}

Let $\theta_L$ and $\theta_R$ be the phases in the left and right superconducting
banks. The Josephson phase is
\begin{equation}
  \varphi_J = \theta_R - \theta_L + \frac{2\pi}{\Phi_0}\,\Phi_{\mathrm{int}},
\end{equation}
where $\Phi_{\mathrm{int}}$ represents the internal flux contribution associated
with the weak-link region. The current--phase relation is
\begin{equation}
  I_s = I_c(d)\sin\varphi_J,
\end{equation}
with $I_c(d)$ depending on the type and thickness of the weak link.

\subsection{Static field distributions and thick SNS regime}

In a thin tunnel junction:
\begin{itemize}
  \item $\theta(\mathbf r)$ changes abruptly across the barrier;
  \item $|\Psi|$ is reduced only in a narrow region;
  \item the local magnetic field associated with the junction is small.
\end{itemize}

In a thick SNS bridge ($d\sim\xi$):
\begin{itemize}
  \item the phase $\theta(x)$ varies smoothly across the normal region;
  \item supercurrents and fields can form extended localized patterns around the
  bridge;
  \item the system may support normal modes localized near the weak link.
\end{itemize}

These localized modes are the classical counterpart of the RF Cells introduced
earlier.

\subsection{RCSJ model and external flux}

The dynamics of the Josephson phase is described by the RCSJ model
\begin{equation}
  C\ddot{\varphi}_J + \frac{1}{R}\dot{\varphi}_J + I_c\sin\varphi_J
  = I_{\mathrm{bias}} + I_{\mathrm{noise}},
\end{equation}
where $C$ is the junction capacitance, $R$ the shunt resistance,
$I_{\mathrm{bias}}$ the bias current, and $I_{\mathrm{noise}}$ a noise term.

A current $I_{\mathrm{coil}}(t)$ in the microcoil produces an external flux
\begin{equation}
  \Phi_{\mathrm{ext}}(t) = M I_{\mathrm{coil}}(t),
\end{equation}
where $M$ is the mutual inductance between the coil and the effective
phase-sensitive loop. Due to screening and geometry, only a fraction $\eta(d)$
of $\Phi_{\mathrm{ext}}$ actually contributes to the phase:
\begin{equation}
  \Phi_{\mathrm{eff}}(t) = \eta(d)\,\Phi_{\mathrm{ext}}(t).
\end{equation}
The phase then acquires an additional term
\begin{equation}
  \varphi_J(t) = \varphi_0(t) +
  \frac{2\pi}{\Phi_0}\,\Phi_{\mathrm{eff}}(t),
\end{equation}
where $\varphi_0(t)$ is the phase in the absence of external flux.

\subsection{Classical definition of $\alpha$ and upper bound}

Comparing with the general linear-response form
\begin{equation}
  \delta\varphi_J(t) =
  \left.\frac{\partial\varphi_J}{\partial\Phi_{\mathrm{ext}}}\right|_{0}
  \delta\Phi_{\mathrm{ext}}(t),
\end{equation}
we identify
\begin{equation}
  \frac{\partial\varphi_J}{\partial\Phi_{\mathrm{ext}}}
  = \frac{2\pi}{\Phi_0}\,\eta(d),
\end{equation}
and thus
\begin{equation}
  \alpha(d) = \frac{\Phi_0}{2\pi}
  \frac{\partial\varphi_J}{\partial\Phi_{\mathrm{ext}}}
  = \eta(d).
\end{equation}

In an ideal SQUID loop with flux $\Phi$ we have
$\varphi_J=(2\pi/\Phi_0)\Phi$, hence $\alpha_{\mathrm{SQUID}}=1$. Any realistic
local-coil geometry without a full loop must satisfy $|\eta(d)|<1$, therefore
\begin{equation}
  0 < |\alpha(d)| \le 1.
\end{equation}
The interval $0<\alpha\le 1$ used in the RF framework is thus naturally
interpreted as a normalization with respect to the SQUID limit.

For typical thin tunnel junctions without special field focusing one expects
$\eta\sim 10^{-3}$--$10^{-2}$, giving
\begin{equation}
  |\alpha_{\mathrm{static}}|\sim 10^{-3}\text{--}10^{-2}
\end{equation}
for the static response, in agreement with the intuition that a single junction
is almost insensitive to a local coil in the absence of a loop.

\subsection{Oscillator estimate for $\alpha(\omega)$}

To capture dynamical effects, we linearize the RCSJ equation around a static
operating point and include the coil drive:
\begin{equation}
  C\ddot{\varphi} + \frac{1}{R}\dot{\varphi} + \frac{1}{L_J}\varphi
  = K I_{\mathrm{coil}}(t),
\end{equation}
where $L_J$ is the Josephson inductance (around the operating point) and $K$ is
a coupling coefficient. The plasma frequency and damping rate are
\begin{equation}
  \omega_p = \sqrt{\frac{1}{L_J C}},\qquad
  \gamma = \frac{1}{2RC}.
\end{equation}

For a harmonic drive $I_{\mathrm{coil}}(t)=I_0 e^{i\omega t}$ the steady-state
solution gives an effective phase response
\begin{equation}
  \alpha_{\mathrm{eff}}(\omega) =
  \frac{\Phi_0}{2\pi}
  \frac{\partial\varphi_J(\omega)}{\partial\Phi_{\mathrm{ext}}(\omega)}
  \propto \frac{1}{\sqrt{(\omega_p^2-\omega^2)^2 + (2\gamma\omega)^2}}.
\end{equation}
In the quasistatic limit $\omega\ll\omega_p$ we have
\begin{equation}
  \alpha_{\mathrm{static}}\equiv \alpha_{\mathrm{eff}}(0)
  \propto \frac{1}{\omega_p^2},
\end{equation}
while at resonance ($\omega=\omega_p$),
\begin{equation}
  \alpha_{\mathrm{res}}\equiv \alpha_{\mathrm{eff}}(\omega_p)
  \propto \frac{1}{2\gamma\omega_p}.
\end{equation}
Eliminating the prefactor yields
\begin{equation}
  \alpha_{\mathrm{res}} \approx \alpha_{\mathrm{static}}\,
  \frac{\omega_p}{2\gamma}
  = \alpha_{\mathrm{static}} Q,
\end{equation}
where $Q=\omega_p/(2\gamma)$ is the quality factor of the mode.

Figure~\ref{fig:alpha-w} shows the corresponding qualitative frequency
dependence. A standard oscillator derivation of this relation is summarized
in Appendix~\ref{app:osc}.

\begin{figure}[ht]
  \centering
  \includegraphics[width=0.7\textwidth]{1b-5.jpg}
  \caption{Schematic frequency dependence of
  $|\alpha_{\mathrm{eff}}(\omega)|$: a small static value in the
  quasistatic regime and a resonant enhancement near $\omega=\omega_p$ by a
  factor of order $Q$.}
  \label{fig:alpha-w}
\end{figure}

\section{Part III: Mapping Between RF and Classical Frameworks}

The preceding sections show that all objects of the RF framework have explicit
counterparts in the classical description. Figure~\ref{fig:mapping} summarizes
this mapping in a compact dictionary.

\begin{figure}[ht]
  \centering
  \small
  \renewcommand{\arraystretch}{1.35}
  \begin{tabular}{
    >{\centering\arraybackslash}p{0.42\textwidth}
    c
    >{\centering\arraybackslash}p{0.42\textwidth}
  }
    \multicolumn{1}{c}{\textbf{Resonant Field framework}} & &
    \multicolumn{1}{c}{\textbf{Classical description}} \\[6pt]
    \midrule
    Phase field $\phi(\mathbf r)$ & $\longleftrightarrow$ &
    Order-parameter phase $\theta(\mathbf r)$ (GL/Usadel) \\[8pt]

    Phase rotator & $\longleftrightarrow$ &
    Weak link with large $\partial\varphi_J/\partial\Phi_{\mathrm{ext}}$ \\[8pt]

    Resonant Field Cell & $\longleftrightarrow$ &
    Localized GL+Maxwell mode (phase, $j_s$, $B$) near the weak link \\[8pt]

    Phase--flux coefficient $\alpha$ & $\longleftrightarrow$ &
    $\alpha(d,\omega) = \dfrac{\Phi_0}{2\pi}
      \left.\dfrac{\partial\varphi_J}{\partial\Phi_{\mathrm{ext}}}\right|_{\omega}$ \\[12pt]

    Phase--Control Equation & $\longleftrightarrow$ &
    Linearized RCSJ with external flux (small-signal phase--flux relation) \\[8pt]

    Strong--rotator regime & $\longleftrightarrow$ &
    Thick SNS, $d\sim\xi$, with
    $\alpha_{\mathrm{res}}\simeq\alpha_{\mathrm{static}}Q\sim 0.3$--$0.6$ \\[4pt]
    \bottomrule
  \end{tabular}

  \caption{Mapping between the Resonant Field framework and the classical
  description. Each RF object (phase field, phase rotator, Resonant Field
  Cell, $\alpha$, the Phase--Control Equation, and the strong-rotator regime)
  has a direct counterpart in the standard GL/Usadel + London--Maxwell +
  RCSJ formulation.}
  \label{fig:mapping}
\end{figure}

This dictionary allows one to use the compact RF language (phase rotators, RF
Cells, $\alpha(d)$) while retaining a clear route back to standard microscopic
theory.

\section{Conclusion}

We have constructed a bridge between the conceptual Resonant Field view of
superconducting weak links and the standard microscopic description based on
GL/Usadel equations, London--Maxwell electrodynamics, and the RCSJ model.

\begin{itemize}
  \item In Part~I we formulated the RF framework: weak links as phase rotators,
  the notion of a Resonant Field Cell, the Phase--Control Equation, and the key
  parameter $\alpha$ that measures how efficiently a local microcoil can rotate
  the Josephson phase.
  \item In Part~II we showed how the same phenomena arise from classical theory:
  the dependence of $I_c(d)$ and phase profiles on thickness, the role of
  localized modes in thick SNS bridges, and the oscillator estimate
  $\alpha_{\mathrm{res}}\approx \alpha_{\mathrm{static}} Q$ for resonant
  enhancement of the phase response.
  \item In Part~III we provided an explicit mapping between RF concepts and
  classical objects, allowing the RF framework to be used as a compact
  high-level language while remaining grounded in established physics.
\end{itemize}

Conceptually, the framework developed here extends the conventional view in
which a strong magnetic influence on the Josephson phase is typically
associated with a macroscopic loop (SQUID geometry). Once localized modes in
thick SNS weak links are taken into account, a single weak link can in
principle reach phase--flux couplings in the range $|\alpha|\sim 0.3$--$0.6$
under resonant driving by a local microcoil.

Potential applications of the strong-rotator regime include: (i) local magnetic
control of Josephson qubits and oscillators without global bias fields,
(ii) ultra-compact SQUID-like magnetometers where the effective phase--flux
coupling is provided by a single thick weak link, and (iii) coil-driven
Josephson oscillators and parametric devices where RF excitation is applied
locally via microcoils rather than global microwave lines. The parameter
$\alpha$ provides a natural figure of merit for the design of such devices.

\appendix

\section{Oscillator derivation of the relation
$\alpha_{\mathrm{res}} \approx \alpha_{\mathrm{static}} Q$}
\label{app:osc}

In this appendix we collect the standard oscillator derivation of the relation
$\alpha_{\mathrm{res}} \approx \alpha_{\mathrm{static}} Q$ used in the main text.

Linearizing the RCSJ equation around a static operating point and including a
small coil current $I_{\mathrm{coil}}(t)$, we obtain
\begin{equation}
  C \ddot{\varphi} +
  \frac{1}{R}\dot{\varphi} +
  \frac{1}{L_J}\,\varphi
  = K I_{\mathrm{coil}}(t),
  \label{eq:appA-osc}
\end{equation}
which is a driven damped harmonic oscillator with natural frequency
\begin{equation}
  \omega_p = \sqrt{\frac{1}{L_J C}}
\end{equation}
and damping rate
\begin{equation}
  \gamma = \frac{1}{2RC},
\end{equation}
so that the quality factor is
\begin{equation}
  Q = \frac{\omega_p}{2\gamma} = \omega_p R C.
\end{equation}

For a harmonic drive
\begin{equation}
  I_{\mathrm{coil}}(t) = I_0 \cos\omega t,
\end{equation}
the steady-state response can be written as
\begin{equation}
  \varphi(t) = \Re\{\tilde{\varphi}(\omega) e^{i\omega t}\},
  \qquad
  \tilde{\varphi}(\omega) =
  \frac{K I_0}{-C(\omega^2 - \omega_p^2) + i (2\gamma C \omega)}.
\end{equation}
The amplitude is therefore
\begin{equation}
  |\tilde{\varphi}(\omega)| =
  \frac{|K| I_0}{\sqrt{(\omega_p^2 - \omega^2)^2 + (2\gamma\omega)^2}}.
\end{equation}

Identifying $\Phi_{\mathrm{ext}} = M I_{\mathrm{coil}}$ and using the definition
of $\alpha_{\mathrm{eff}}(\omega)$ via
\begin{equation}
  \delta\varphi(\omega) =
  \alpha_{\mathrm{eff}}(\omega)\,
  \frac{2\pi}{\Phi_0}\,\Phi_{\mathrm{ext}}(\omega)
  = \alpha_{\mathrm{eff}}(\omega)\,
  \frac{2\pi}{\Phi_0}\,M I_0,
\end{equation}
we obtain
\begin{equation}
  \alpha_{\mathrm{eff}}(\omega) =
  \frac{|\delta\varphi(\omega)|}{(2\pi/\Phi_0) M I_0}
  = \frac{|K|\Phi_0}{2\pi M}
  \frac{1}{\sqrt{(\omega_p^2 - \omega^2)^2 + (2\gamma\omega)^2}}.
\end{equation}

In the quasistatic limit $\omega \ll \omega_p$ we have
\begin{equation}
  \alpha_{\mathrm{static}} \equiv \alpha_{\mathrm{eff}}(0) \approx
  \frac{|K|\Phi_0}{2\pi M}\,
  \frac{1}{\omega_p^2},
\end{equation}
while at resonance ($\omega = \omega_p$),
\begin{equation}
  \alpha_{\mathrm{res}} \equiv \alpha_{\mathrm{eff}}(\omega_p) =
  \frac{|K|\Phi_0}{2\pi M}\,
  \frac{1}{2\gamma\omega_p}.
\end{equation}
Eliminating the prefactor via $\alpha_{\mathrm{static}}$ yields
\begin{equation}
  \alpha_{\mathrm{res}} \approx
  \alpha_{\mathrm{static}}\,
  \frac{\omega_p}{2\gamma}
  = \alpha_{\mathrm{static}} Q,
\end{equation}
which is the relation used in Sec.~3.7 and in the RF framework discussion.

\section{Order-of-magnitude estimate for a thick SNS bridge with microcoil}

The oscillator relation
$\alpha_{\mathrm{res}} \approx \alpha_{\mathrm{static}} Q$
shows that a modest static coupling $\alpha_{\mathrm{static}}$ can be amplified
into a strong-rotator regime at resonance, provided the mode has a reasonable
quality factor $Q$. Here we collect a simple numerical estimate for a thick SNS
bridge addressed by an on-chip microcoil.

Consider a diffusive SNS weak link with
\begin{itemize}
  \item normal-layer thickness $d \sim 10~\mathrm{nm}$,
  \item width $w \sim 300~\mathrm{nm}$,
  \item coherence length $\xi$ such that $d\sim\xi$ (thick SNS regime),
  \item London penetration depth in the electrodes
        $\lambda_L \sim 250~\mathrm{nm}$.
\end{itemize}
For such dimensions the effective kinetic inductance of the weak-link region
can be of order
\begin{equation}
  L_{\mathrm{kin}} \sim 20\text{--}30~\mathrm{pH},
\end{equation}
while realistic planar microcoils placed above the bridge can reach mutual
inductances in the range
\begin{equation}
  M \sim 0.8\text{--}1.8~\mathrm{pH}.
\end{equation}

In a simple lumped picture the static phase--flux coupling can be estimated as
\begin{equation}
  \alpha_{\mathrm{static}}
  \sim \frac{M}{L_{\mathrm{kin}}}
  \sim 0.03\text{--}0.07,
\end{equation}
which already lies two--three orders of magnitude above the tunnel-junction
values $|\alpha_{\mathrm{static}}|\sim 10^{-3}$--$10^{-2}$ quoted in the main
text for typical SIS geometries.

For localized plasma-like modes in the $5$--$20~\mathrm{GHz}$ range, quality
factors of order
\begin{equation}
  Q \sim 15\text{--}35
\end{equation}
are not unreasonable for carefully engineered on-chip structures. Combining the
two estimates, we obtain a resonant phase--flux coupling
\begin{equation}
  |\alpha_{\mathrm{res}}|
  \sim |\alpha_{\mathrm{static}}|\,Q
  \sim 0.3\text{--}0.6,
\end{equation}
which falls squarely into the target strong-rotator window of the RF
framework.

This back-of-the-envelope estimate does not replace full GL/Usadel +
London--Maxwell simulations, but it shows that thick SNS bridges with realistic
microcoils can, in principle, reach the strong-rotator regime without
requiring extreme values of $M$, $L_{\mathrm{kin}}$, or $Q$. It also motivates
the experimental programme described in Ref.~\cite{Experiment}, where
coil-driven Shapiro steps in such devices are proposed as a direct operational
probe of $\alpha(d,\omega)$.

\section*{Acknowledgements}

The author thanks colleagues and collaborators for discussions on SNS weak
links, localized modes and resonant control of Josephson phase.

\begin{thebibliography}{9}

\bibitem{PartI}
A.~Turchanov,
\newblock \emph{Resonant Field Symmetry in Superconductors: A Standing-Wave
Picture of Meissner Screening and Josephson Barriers},
Phase Geometry Series --- Part I, preprint (2025).

\bibitem{PartII}
A.~Turchanov,
\newblock \emph{Phase-Coherent Josephson Devices as Clocks in Weak Gravitational
Fields},
Phase Geometry Series --- Part II, preprint (2025).

\bibitem{PartIII}
A.~Turchanov,
\newblock \emph{Phase-Field Newtonian Gravity and Phase Clocks},
Phase Geometry Series --- Part III, Zenodo (2025).

\bibitem{Experiment}
A.~Turchanov,
\newblock \emph{Magnetic Phase Control of a Thick SNS Weak Link},
experimental proposal / preprint (2025).

\end{thebibliography}

\end{document}
