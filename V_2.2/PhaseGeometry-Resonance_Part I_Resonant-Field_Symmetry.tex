\documentclass[11pt,a4paper]{article}

\usepackage[margin=2.5cm]{geometry}
\usepackage{amsmath,amssymb}
\usepackage{graphicx}
\usepackage{bm}
\usepackage{hyperref}
\usepackage{cite}
\usepackage{titling} % добавлено для управления позицией заголовка

\setlength{\droptitle}{-1cm} % поднять блок заголовка примерно на 1 см вверх

\title{PhaseGeometry-Resonance --- Part I\\[4pt]
Resonant Field Symmetry in Superconductors:\\
A Standing-Wave Picture of Meissner Screening and Josephson Barriers}

\author{Aleksey Turchanov}
\date{November 2025\\[0.3em]
\scriptsize Licensed under CC BY 4.0. Zenodo DOI: 10.5281/zenodo.17807163}

\begin{document}

\maketitle

\begin{abstract}
We propose a phase--geometric view of superconductivity that we call
\emph{resonant field symmetry}. Instead of treating the Meissner effect,
surface screening currents and Josephson weak links as separate ingredients,
we model a superconductor as a phase--coherent medium in which electric and
magnetic field components form standing structures. In this picture,
Meissner screening is understood not as the disappearance of magnetic field
inside the sample, but as a rearrangement of a standing wave and surface
currents such that the \emph{macroscopic} magnetic flux through the
superconductor vanishes, while field energy remains localized within a thin
surface layer of thickness $\lambda_L$.

Within this language a Josephson barrier or a thick SNS bridge is naturally
described as a \emph{phase rotator}: a local region in which the phase flow
is turned by a finite angle, and part of the energy can be stored in a
localized magnetic configuration around the weak link. The barrier thickness
then controls the balance between purely tunneling transport and local
magnetic ``tension'' in this region.

The formalism is meant as a compact effective language on top of the
standard Ginzburg--Landau, London--Maxwell and Josephson theory, not as an
alternative microscopic model. In the companion paper
\emph{Resonant Field Framework and Classical Basis}
(Phase Geometry Series --- Part I-B) the same objects are introduced in a
more technical way, and the phase--flux sensitivity of a weak link is
described by a normalized coefficient
\(
\alpha(d,\omega) = (\Phi_0/2\pi)\, \partial\varphi_J/\partial\Phi_{\text{ext}}
\),
computed within the classical GL/Usadel, London--Maxwell and RCSJ framework
for a concrete geometry: a thick SNS weak link driven by a local microcoil.
Here we focus instead on an intuitive standing-wave picture of Meissner
screening, Josephson barriers and thick weak links as phase rotators with
localized resonant field cells.

\medskip
\noindent\textbf{Standard.}
All ingredients remain within conventional Ginzburg--Landau theory,
London--Maxwell electrodynamics and Josephson phenomenology; no microscopic
BCS assumptions are modified, and all observable consequences are consistent
with standard superconductivity.

\noindent\textbf{New in this note.}
A phenomenological standing--wave picture of Meissner screening, the notion
of phase rotators and localized resonant field cells, and a qualitative map
between barrier thickness, phase--flux sensitivity $\alpha(d,\omega)$ and
possible microcoil-based phase-control experiments.
\end{abstract}
\pagebreak
\tableofcontents

\section{Introduction}

The conventional description of superconductors is usually built as a set of
separate blocks: London equations for screening currents, Josephson
phenomenology for weak links, models of tunneling barriers, and SQUID
circuit theory. In this standard approach, Meissner screening, supercurrents
and Josephson phase dynamics appear as different effects that must be
manually ``glued together'' when one considers a specific device or
geometry.

In this work we propose a phenomenological, more geometric reinterpretation,
which we call \emph{resonant field symmetry}. Instead of treating the
magnetic field, currents and barriers separately, we view a superconductor
as a phase--coherent medium in which the electromagnetic field can form a
standing structure. Meissner ``expulsion'' is then understood not as the
absence of field, but as a rearrangement of the standing wave and screening
currents such that the \emph{net} flux through the sample is zero.

On this level it is more convenient to speak not about ``magnetic field
inside'' and ``current at the surface'', but about a phase field
$\phi(\mathbf{r})$ that encodes the standing structure. A weak link (a
Josephson tunnel junction, an SNS bridge, a nanobridge) is then naturally
described as a \emph{phase rotator}: a local region in which the phase flow
is turned by a finite angle $\Delta\phi$, and part of the energy can be
stored in a localized magnetic configuration around the weak link. The
thickness of the barrier controls the balance between purely tunneling
transport and local magnetic ``tension'' in this region.

This language is practically useful because it compresses a complicated
configuration of fields and currents into a small number of geometric
objects: a phase field $\phi(\mathbf{r})$, rotators (places where the phase
is turned), and localized resonant regions --- \emph{Resonant Field Cells}
--- where the field and phase form a standing pattern. For device design
this is often enough: one can mentally assemble a ``phase circuit'' from
rotators and standing segments and immediately see where the system stores
energy and where it efficiently transmits phase.

At the same time, the present language is not meant to replace microscopic
BCS theory or the standard Ginzburg--Landau framework. It is intended as a
compact \emph{upper layer} above the familiar theory. In the companion paper
\emph{Resonant Field Framework and Classical Basis}
(Phase Geometry Series --- Part I-B) the same objects are introduced more
formally: a phase rotator is identified with a weak link whose normalized
phase--flux response
\[
\alpha(d,\omega)
= \frac{\Phi_0}{2\pi}\,
  \frac{\partial\varphi_J}{\partial\Phi_{\text{ext}}}\Big|_\omega
\]
is expressed in terms of GL/Usadel, London--Maxwell and RCSJ parameters.
There we show, for a particular geometry of a thick SNS weak link with a
local microcoil, how one can estimate $\alpha(d,\omega)$ and identify a
strong--rotator regime with $|\alpha|\sim 0.3$--$0.6$.

The present text should therefore be read as a \emph{manifest} and an
intuitive introduction to a phase--geometric view of superconductivity. We
deliberately use simplified schematics and qualitative reasoning: the goal
is to understand how Meissner screening, a Josephson barrier and a thick
bridge look if one thinks in terms of standing phase structures and local
rotators. Readers who need detailed calculations of $\alpha(d,\omega)$,
critical currents or Shapiro step heights can use the companion Part I-B as
a ``working dictionary'', where the resonant field picture is tied to the
standard classical theory.

In what follows we proceed as follows. In Sec.~\ref{sec:standing-meissner}
we discuss wave geometry and standing symmetry in a simple superconducting
cylinder and reinterpret Meissner screening in terms of a standing $B$-field
mode and surface currents. In Sec.~\ref{sec:rotator} we discuss the
Josephson barrier as a phase rotator. In Sec.~\ref{sec:thick-cell} we
consider thick weak links and the qualitative emergence of a localized
resonant field cell around a bridge. In Sec.~\ref{sec:alpha-d} we sketch the
expected qualitative dependence of the phase--flux coefficient $\alpha(d)$
on barrier thickness and the transition to a strong--rotator regime. In
Sec.~\ref{sec:experiments} we outline possible phase-control experiments
with local microcoils. Finally, in Sec.~\ref{sec:outlook} we discuss how
Manifest I, its Part I-B companion and subsequent Parts II and III of the
Phase Geometry Series fit together into a broader phase--geometric framework
that also covers phase clocks and weak gravitational fields.

\section{Standing-wave symmetry and Meissner screening}
\label{sec:standing-meissner}

We begin with a simple textbook geometry: a long superconducting cylinder in
an external magnetic field. In the conventional picture, the London
equations imply that the field is expelled from the bulk, decaying on the
length scale of the London penetration depth $\lambda_L$, while surface
supercurrents $j_s$ flow in a thin layer near the boundary. The Meissner
effect is then often phrased as ``the superconductor expels magnetic
field''. 

In the resonant field picture, we instead think of the cylinder as hosting a
standing structure of the electromagnetic field. In the Meissner state the
macroscopic flux through the cross-section vanishes, but the field energy is
concentrated in a thin surface layer of thickness $\lambda_L$, where the
magnetic field $B$ and the surface current $j_s$ form a resonant pattern.

\begin{figure}[t]
  \centering
  \includegraphics[width=0.6\textwidth]{1-1.jpg}
  \caption{Standing-wave symmetry in a superconducting cylinder.
  Two counter-propagating components of the magnetic field $B$ form a
  standing mode inside the sample, while the macroscopic flux is cancelled
  by a surface supercurrent $j_s$ flowing in a layer of thickness
  $\lambda_L$. The picture is schematic: it is meant to emphasize the
  standing-wave interpretation rather than provide an exact field profile.}
  \label{fig:standing-wave-cylinder}
\end{figure}

Schematically, as shown in Fig.~\ref{fig:standing-wave-cylinder}, we
represent this as two counter-propagating $B$-waves along the cylinder axis,
forming a standing mode, and a surface current $j_s$ flowing along the
boundary within a layer of thickness $\lambda_L$. The emphasis is not on the
exact solution of Maxwell's equations for this geometry, but on the
interpretation: the Meissner state is viewed as a standing-wave symmetry
between internal field energy and surface current, arranged in such a way
that the net flux through the sample vanishes.

From this point of view, a superconducting body is a \emph{phase resonator}.
The Meissner effect reflects the ability of the system to adjust its
standing phase and field structure to minimize the macroscopic magnetic
energy while keeping phase coherence and field energy stored in a thin
surface layer.

% TODO: further qualitative discussion of standing modes, phase field ϕ(r),
% and how standard London results can be rephrased in this language.

\section{Josephson barriers as phase rotators}
\label{sec:rotator}

In the standard Josephson picture, a weak link between two superconducting
banks supports a current--phase relation
\begin{equation}
  I_s = I_c \sin\varphi_J,
\end{equation}
where $\varphi_J$ is the phase difference between the order parameters on
the two sides of the barrier. The barrier thickness and type (SIS, SNS,
nanobridge) determine the critical current $I_c$ and the detailed form of
the current--phase relation.

In the resonant field language we instead focus on how the phase field
$\phi(\mathbf{r})$ is \emph{turned} by the barrier. A Josephson weak link is
pictured as a \emph{phase rotator}: a localized region where the phase flow
changes by a finite angle $\Delta\phi$, and where the standing field
structure is allowed to distort and store energy.

Qualitatively, a very thin tunnel barrier corresponds to a sharp phase jump
concentrated in a narrow region, with negligible internal field structure:
the rotator is ``stiff''. As the barrier becomes thicker (for instance, in
an SNS bridge with thickness $d\sim\xi$), the phase drop is distributed over
a larger region and a localized pattern of currents and fields can develop
around the weak link. In this regime the weak link behaves as a more
``flexible'' rotator that can store field energy in a localized resonant
field cell.

% TODO: discuss qualitative phase profiles across thin vs thick barriers,
% and how the rotator picture interpolates between tunneling-dominated and
% field-storage-dominated regimes.

\section{Thick weak links and localized Resonant Field Cells}
\label{sec:thick-cell}

A key qualitative prediction of the resonant field picture is that thick
weak links (for example, SNS bridges with normal-layer thickness $d$ of
order the coherence length $\xi$) can host localized field structures that
are much more responsive to external excitation than thin tunnel junctions.

To make this concrete we consider a thin superconducting strip with a
central thick weak link and a local microcoil placed above the weak-link
region, as in Fig.~\ref{fig:thick-weak-link-geometry}. A DC bias current
$I_{\mathrm{DC}}$ flows along the strip, while an AC current $I_{\mathrm{AC}}$
in the microcoil produces a time-dependent magnetic field $B_{\mathrm{AC}}(t)$
localized near the weak link.

\begin{figure}[t]
  \centering
  \includegraphics[width=0.8\textwidth]{1-2.jpg}
  \caption{Schematic geometry: a superconducting strip with a central thick
  weak link and a local microcoil above it. A DC current $I_{\mathrm{DC}}$
  flows along the strip, while an AC current $I_{\mathrm{AC}}$ in the coil
  generates a local magnetic field $B_{\mathrm{AC}}(t)$ above the weak-link
  region. The weak link acts as a phase rotator embedded in a localized
  Resonant Field Cell.}
  \label{fig:thick-weak-link-geometry}
\end{figure}

In the resonant field language the weak link plus its surrounding field
structure are treated as a compact object: a phase rotator surrounded by a
Resonant Field Cell. The cell can support a quasi-standing mode of current
and field, which couples the external drive from the coil to the phase
across the weak link. The strength of this coupling is characterized by a
dimensionless phase--flux coefficient $\alpha$, defined at the effective
level by
\begin{equation}
  \delta\varphi_{\mathrm{mag}}(t)
  = \alpha\,\frac{2\pi}{\Phi_0}\,\Phi_{\mathrm{AC}}(t),
\end{equation}
where $\Phi_{\mathrm{AC}}(t)$ is the effective flux seen by the weak link,
and $\delta\varphi_{\mathrm{mag}}(t)$ is the magnetic contribution to the
Josephson phase.

Thin junctions typically have extremely small $|\alpha|$, so that a local
coil produces negligible phase modulation. Thick weak links with a well
developed Resonant Field Cell, on the other hand, can in principle reach
$|\alpha|\sim 0.1$--$1$ under resonant excitation, making them strong phase
rotators controllable by local microcoils.

% TODO: discuss qualitatively how the RF Cell emerges as d increases, and how
% energy is shared between phase flow and local magnetic structure.

\section{Qualitative dependence $\alpha(d)$ and the strong--rotator regime}
\label{sec:alpha-d}

From the resonant field viewpoint the dependence of $|\alpha|$ on the
thickness $d$ of the weak link is expected to be non-monotonic:

\begin{itemize}
  \item For very thin barriers ($d\ll\xi$), the phase drop is sharp and the
  internal field structure is negligible; the weak link behaves as a very
  weak rotator with $|\alpha(d)|\ll 10^{-2}$.

  \item For $d\sim\xi$ (thick SNS bridges), the phase drop is spread over
  a wider region, a localized standing pattern of currents and fields can
  form, and the coupling to a local microcoil can increase dramatically.
  In this regime a well defined Resonant Field Cell appears around the weak
  link, and $|\alpha(d)|$ can reach values of order $0.1$--$1$ under
  resonant drive.

  \item For very large $d\gg\xi$, the superconducting coupling across the
  weak link is degraded, the cell becomes diffuse and lossy, and $|\alpha(d)|$
  is expected to decrease again.
\end{itemize}

Qualitatively, one can introduce a target range for a \emph{strong rotator}
regime,
\begin{equation}
  |\alpha_{\mathrm{target}}|\sim 0.3\text{--}0.6,\qquad d\sim\xi,
\end{equation}
corresponding to a two-- to three--order-of-magnitude enhancement of local
phase sensitivity compared to the tunnel limit. The detailed shape of
$\alpha(d)$ and its frequency dependence $\alpha(d,\omega)$ must eventually
be obtained from microscopic calculations or from experiment. In the
companion Part I-B the same picture is developed within the classical
GL/Usadel + RCSJ framework, and a simple oscillator estimate
$\alpha_{\mathrm{res}}\approx\alpha_{\mathrm{static}}Q$ is derived for
resonant enhancement of the phase response.

% TODO: include a simple schematic plot of |α(d)| vs d/ξ if desired.

\section{Possible phase-control experiments with local microcoils}
\label{sec:experiments}

The resonant field picture suggests a simple and direct experimental test:
take a thick SNS weak link (with $d\sim\xi$) embedded in a superconducting
strip, place a microcoil above it, and measure coil-driven phase modulation
via Shapiro steps or other phase-sensitive signatures.

In the standard Josephson framework, Shapiro steps on the $I$--$V$
characteristic arise when the phase is driven by an AC voltage at frequency
$\omega$, leading to phase locking and quantized voltage plateaus. In the
present context, the coil acts as a \emph{magnetic} drive: an AC current
$I_{\mathrm{AC}}(t)$ generates an AC flux $\Phi_{\mathrm{AC}}(t)$, which
modulates the phase via the phase--flux coupling characterized by $\alpha$.
In a weak rotator ($|\alpha|\ll10^{-2}$), coil-only drive produces
negligible Shapiro steps; in a strong rotator ($|\alpha|\sim 0.3$--$0.6$),
the same drive produces clearly visible steps, even without a direct RF
voltage on the junction.

An experiment of this type would provide a direct measurement of $\alpha(d)$
for different weak-link thicknesses, and would test the qualitative
prediction that thick SNS bridges can act as strong phase rotators. A more
detailed design and analysis of such an experiment is the subject of the
companion device-oriented work \emph{Magnetic Phase Control of a Thick SNS
Weak Link}, which builds on both the present Manifest I and the classical
framework of Part I-B.

% TODO: outline specific measurement protocols if desired.
\section{Outlook and relation to the Phase Geometry Series}
\label{sec:outlook}

In this Manifest I we have introduced the idea of \emph{resonant field
symmetry} as an intuitive, standing-wave picture of superconducting
screening and weak links. Superconductors are treated as phase resonators,
Josephson barriers as phase rotators, and thick weak links as centers of
localized Resonant Field Cells that can strongly couple local magnetic
excitation to the Josephson phase.

The present work is deliberately phenomenological and device-oriented. The
companion paper \emph{Resonant Field Framework and Classical Basis}
(Phase Geometry Series --- Part I-B) develops the same ideas in a more
technical form for a concrete system: a thick SNS weak link driven by a
local microcoil. There the phase--flux coefficient $\alpha(d,\omega)$ is
defined as a calculable response function within the GL/Usadel,
London--Maxwell and RCSJ framework, and the strong--rotator regime
$|\alpha|\sim 0.3$--$0.6$ is identified as a realistic target for resonant
SNS structures.

In \emph{Phase Geometry Series --- Part II} we take a different step: the
macroscopic phase is embedded into a five-dimensional phase-fibre geometry
of Kaluza--Klein type, and used as a time-keeping variable for
\emph{phase clocks} in weak gravitational fields. There we derive a
phase-rate law $d\varphi/dt = \omega_0\sqrt{-g_{00}}$ and apply it to the
gravitational redshift of Josephson frequencies and rotation-induced phase
shifts in SQUIDs. In \emph{Phase Geometry Series --- Part III} we further
develop a phase-field version of Newtonian gravity, where gradients of a
phase field act as an effective mass density sourcing the usual Poisson
equation, and phase clocks probe the resulting potential.

Together, Manifest I, its Part I-B companion and Parts II--III of the Phase
Geometry Series provide three complementary layers:
a phenomenological standing-wave picture of superconducting structures,
a classical GL/London/RCSJ dictionary for phase rotators and resonant cells,
and a phase-geometric framework that connects macroscopic coherence to weak
gravitational and inertial effects.


\begin{itemize}
  \item an \emph{intuitive} standing-wave and rotator picture of
  superconducting coherence (Manifest I);
  \item a \emph{calculational} framework that ties this picture to the
  standard GL/Usadel + London--Maxwell + RCSJ theory and to measurable
  quantities such as $\alpha(d,\omega)$ (Part I-B);
  \item a more \emph{geometric} viewpoint in which phase and electromagnetic
  structure are embedded into an extended spacetime-like geometry used for
  phase clocks and weak gravity (Parts II and III).
\end{itemize}

The goal is not to replace the existing theories, but to provide a coherent
phase--geometric language that makes it easier to think about, design and
generalize superconducting devices and phase-based clocks. The resonant
field symmetry introduced here is the first step in that direction.

\section*{Acknowledgements}

% (Placeholder for acknowledgements)

\begin{thebibliography}{9}

\bibitem{PartIB}
A.~Turchanov, \emph{Resonant Field Framework and Classical Basis for
Magnetic Phase Control of a Thick SNS Weak Link}, Phase Geometry Series ---
Part I-B, preprint (2025).

\bibitem{PartII}
A.~Turchanov, \emph{Phase-Coherent Josephson Devices as Clocks in Weak
Gravitational Fields}, Phase Geometry Series --- Part II, preprint (2025).

\bibitem{PartIII}
A.~Turchanov, \emph{Phase-Field Newtonian Gravity and Phase Clocks},
Phase Geometry Series --- Part III, preprint (2025).

\end{thebibliography}

\end{document}
